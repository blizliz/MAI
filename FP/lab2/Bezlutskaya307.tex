\documentclass[12pt]{article}

\usepackage{fullpage}
\usepackage{multicol,multirow}
\usepackage{tabularx}
\usepackage{ulem}
\usepackage[utf8]{inputenc}
\usepackage[russian]{babel}
\usepackage{amsmath}
\usepackage{amssymb}
\usepackage{listing}

\usepackage{titlesec}

\titleformat{\section}
  {\normalfont\Large\bfseries}{\thesection.}{0.3em}{}

\titleformat{\subsection}
  {\normalfont\large\bfseries}{\thesubsection.}{0.3em}{}

\titlespacing{\section}{0pt}{*2}{*2}
\titlespacing{\subsection}{0pt}{*1}{*1}
\titlespacing{\subsubsection}{0pt}{*0}{*0}
\usepackage{listings}
\lstloadlanguages{Lisp}
\lstset{extendedchars=false,
	breaklines=true,
	breakatwhitespace=true,
	keepspaces = true,
	tabsize=2
}
\begin{document}


\section*{Отчет по лабораторной работе №\,2 
по курсу \guillemotleft  Функциональное программирование\guillemotright}
\begin{flushright}
Студент группы М8О-307 МАИ \textit{Безлуцкая Елизавета}, \textnumero 2 по списку \\
\makebox[7cm]{Контакты: {\tt lizabezlutskaya@gmail.com} \hfill} \\
\makebox[7cm]{Работа выполнена: 28.03.2019 \hfill} \\
\ \\
Преподаватель: Иванов Дмитрий Анатольевич, доц. каф. 806 \\
\makebox[7cm]{Отчет сдан: \hfill} \\
\makebox[7cm]{Итоговая оценка: \hfill} \\
\makebox[7cm]{Подпись преподавателя: \hfill} \\

\end{flushright}

\section{Тема работы}
Простейшие функции работы со списками Коммон Лисп.

\section{Цель работы}
Научиться конструировать списки, находить элемент в списке, использовать схему линейной и древовидной рекурсии для обхода и реконструкции плоских списков и деревьев.

\section{Задание (вариант №11)}
Дан список действительных чисел ($x_{1}$ ... $x_{n}$), \(n \geqslant 2\).
Запрограммируйте рекурсивно на языке Коммон Лисп функцию, вычисляющую выражение вида:\\

$(x_{1} + x_{n}) * (x_{2} + x_{n-1}) * ... * (x_{n} + x_{1})$.

\section{Оборудование студента}
Процессор Intel Core i5-3230M 4\,@\,2.6GHz, память: 350Gb, разрядность системы: 64.

\section{Программное обеспечение}
Ubuntu 16.04 LTS, clisp compiler

\section{Идея, метод, алгоритм}
Функция {\tt product-sum2} вызывает вспомогательную функцию {\tt sub-product-sum2} cо следующими аргументами: списком {\tt l} и перевёрнутым списком {\tt l}. {\tt sub-product-sum2} рекурсивна и работает следующим образом:
\begin{itemize}
\setlength{\itemsep}{-1mm} % уменьшает расстояние между элементами списка
\item если список {\tt l} пуст, функция вернет 1, иначе:
\item перемножаем сумму первых элементов списков и результат рекурсивного вызова функции {\tt sub-product-sum2} с аргументами в виде хвостов текущих списков.
\end{itemize}

\section{Сценарий выполнения работы}

\section{Распечатка программы и её результаты}

\subsection{Исходный код}

\begin{lstlisting}
(defun sub-product-sum (l1 l2)
    (cond ((null l1) 1)
          (t (* (+ (first l1)(first l2))(sub-product-sum (rest l1)(rest l2))))))

(defun product-sum2 (l)
    (sub-product-sum l (reverse l)))
\end{lstlisting}

\subsection{Результаты работы}
\begin{lstlisting}
(print (product-sum2 '(1 2 3 4 5)))
7776
(print (product-sum2 '(0 0)))
0
(print (product-sum2 '(1 -1)))
0
(print (product-sum2 '(1 0 1 1)))
4
(print (product-sum2 '(-8 5)))
9
(print (product-sum2 '(-8 -5)))
169
(print (product-sum2 '(-8 -5 5)))
-90
(print (product-sum2 '(1 3 -5 4 6 8 -2 4 9 4 0)))
12845056 
\end{lstlisting}

%\section{Дневник отладки}
%\begin{tabular}{|c|p{5cm}|p{5cm}|c|}
%\hline
%Дата & Событие & Действие по исправлению & Примечание \\
%\hline
%\end{tabular}

\section{Замечания автора по существу работы}
С хвостовой рекурсией я познакомилась, когда проходила курс <<Логическое программирование>> и писала код лабораторных работ на языке Prolog. В связи с этим трудностей у меня не возникло.

\section{Выводы}
Первоначально задуманное решение не требовало дополнительной памяти, но требовало отсечения последнего элемента списка. Варианты реализации такой функции мне показались не очень привлекательными, поэтому я изменила тактику.
В итоге программа работает за линейное время и память. 
\end{document}\grid
\grid
