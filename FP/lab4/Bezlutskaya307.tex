\documentclass[12pt]{article}

\usepackage{fullpage}
\usepackage{multicol,multirow}
\usepackage{tabularx}
\usepackage{ulem}
\usepackage[utf8]{inputenc}
\usepackage[russian]{babel}
\usepackage{amsmath}
\usepackage{amssymb}
\usepackage{listing}
\usepackage{graphicx}
\usepackage{color}
\usepackage{titlesec}

\titleformat{\section}
  {\normalfont\Large\bfseries}{\thesection.}{0.3em}{}

\titleformat{\subsection}
  {\normalfont\large\bfseries}{\thesubsection.}{0.3em}{}

\titlespacing{\section}{0pt}{*2}{*2}
\titlespacing{\subsection}{0pt}{*1}{*1}
\titlespacing{\subsubsection}{0pt}{*0}{*0}
\usepackage{listings}
\lstloadlanguages{Lisp}
\lstset{extendedchars=false,
	breaklines=true,
	breakatwhitespace=true,
	keepspaces = true,
	tabsize=2
}
\begin{document}


\section*{Отчет по лабораторной работе №\,4 
по курсу \guillemotleft  Функциональное программирование\guillemotright}
\begin{flushright}
Студент группы М8О-307 МАИ \textit{Безлуцкая Елизавета}, \textnumero 2 по списку \\
\makebox[7cm]{Контакты: {\tt lizabezlutskaya@gmail.com} \hfill} \\
\makebox[7cm]{Работа выполнена: 23.05.2019 \hfill} \\
\ \\
Преподаватель: Иванов Дмитрий Анатольевич, доц. каф. 806 \\
\makebox[7cm]{Отчет сдан: \hfill} \\
\makebox[7cm]{Итоговая оценка: \hfill} \\
\makebox[7cm]{Подпись преподавателя: \hfill} \\

\end{flushright}

\section{Тема работы}
Знаки и строки.

\section{Цель работы}
Научиться работать с литерами (знаками) и строками при помощи функций обработки строк и общих функций работы с последовательностями.

\section{Задание (вариант №36)}
Запрограммировать на языке Коммон Лисп функцию, принимающую один аргумент - текст.\\

Если в тексте нет малых букв, функция должна вернуть этот текст без изменения. В противном случае функция должна вернуть копию текста, в котором после всех слов, состоящих из малых букв, вставлен знак пунктуации , (запятая).\\

Функция должна работать как для малых латинских, так и малых русских букв.

\section{Оборудование студента}
Процессор Intel Core i5-3230M 4\,@\,2.6GHz, память: 16Gb, разрядность системы: 64.

\section{Программное обеспечение}
Ubuntu 16.04 LTS, clisp compiler

\section{Идея, метод, алгоритм}
Основная функция {\tt text-processing} вызывает вспомогательную функцию \\
{\tt sentence-processing}, которая обрабатывает каждое предложение текста по отдельности.\\

Функция {\tt sentence-processing} работает следующим образом:
\begin{itemize}
\setlength{\itemsep}{-1mm} % уменьшает расстояние между элементами списка
\item если встретилось слово, в нём проверяется наличие знака нижнего региста
\begin{itemize}
\item встретился хоть один знак нижнего регистра -- в возвращаемую коллекцию добавляется данное слово плюс знак запятой после слова
\item иначе -- слово без изменений
\end{itemize}
\item если встретились знаки-разделители(один или более) -- в возвращаемую коллекцию добавляется вся последовательность разделителей
\end{itemize}

Результат обработки предложения должен быть строкой, а функция {\tt sentence-processing} возвращает последовательность строк. Для преобразования <<последовательность строк-одна строка>> вводится функция {\tt concat-string}.


\section{Сценарий выполнения работы}

\section{Распечатка программы и её результаты}

\subsection{Исходный код}

\definecolor{codegreen}{rgb}{0,0.6,0}
\definecolor{codegray}{rgb}{0.5,0.5,0.5}
\definecolor{codepurple}{rgb}{0.58,0,0.82}
\definecolor{backcolour}{rgb}{0.95,0.95,0.92}
 
\lstdefinestyle{mystyle}{
    backgroundcolor=\color{backcolour},   
    commentstyle=\color{codegreen},
    keywordstyle=\color{magenta},
    numberstyle=\tiny\color{codegray},
    stringstyle=\color{codepurple},
    basicstyle=\footnotesize,
    breakatwhitespace=false,         
    breaklines=true,                 
    captionpos=b,                    
    keepspaces=true,                 
    numbers=left,                    
    numbersep=5pt,                  
    showspaces=false,                
    showstringspaces=false,
    showtabs=false,                  
    tabsize=2
}
 
\lstset{style=mystyle} 

\begin{lstlisting}
(defun whitespace-char-p (char)
  (member char '(#\Space #\Tab #\Newline)))

(defun russian-lower-case-p (char)
  (position char "абвгдеёжзиклмнопрстуфхцчшщъыьэюя"))

(defun text-processing (text)
    (loop for sentence in text
        collect (concat-string (sentence-processing sentence))))

(defun sentence-processing (string)
    (loop with len = (length string)
        for left = 0 then (or (position-if-not #'whitespace-char-p string :start right) len)
        for right = (or (position-if #'whitespace-char-p string :start left) len)

        if (or (some #'lower-case-p (subseq string left right)) 
               (some #'russian-lower-case-p (subseq string left right)))
            collect (concatenate 'string (subseq string left right) ",")
        if (not (or (some #'lower-case-p (subseq string left right)) 
                    (some #'russian-lower-case-p (subseq string left right))))
            collect (subseq string left right)

        collect (subseq string right 
            (or (position-if-not #'whitespace-char-p string :start right) len))
        
        while (< right len)))

(defun concat-string (list)
  "A non-recursive function that concatenates a list of strings."
  (if (listp list)
      (let ((result ""))
        (dolist (item list)
          (if (stringp item)
              (setq result (concatenate 'string result item))))
        result)))
        result)))
\end{lstlisting}


\subsection{Результаты работы}
\begin{lstlisting}
(setq txt '("А СУДЬИ КТо? – За древностию лет"

"К свободной жизни их вражда непримирима,
Сужденья черпают из забытых газет
Времён Очаковских и покоренья Крыма;"))

(print (text-processing txt))

("А СУДЬИ КТо?, – За, древностию, лет,"
 "К свободной, жизни, их, вражда, непримирима,,
Сужденья, черпают, из, забытых, газет,
Времён, Очаковских, и, покоренья, Крыма;,") 


(setq txt '("ГДЕ? УКАЖИТЕ НАМ, ОТЕЧЕСТВА ОТЦЫ,"

"КОТОРЫХ МЫ ДОЛЖНЫ ПРИНЯТЬ ЗА ОБРАЗЦЫ?"))

(print (text-processing txt))

("ГДЕ? УКАЖИТЕ НАМ, ОТЕЧЕСТВА ОТЦЫ," "КОТОРЫХ МЫ ДОЛЖНЫ ПРИНЯТЬ ЗА ОБРАЗЦЫ?") 
\end{lstlisting}

\section{Дневник отладки}
\begin{tabular}{|c|p{5cm}|p{5cm}|p{3cm}|}
\hline
Дата & Событие & Действие по исправлению & Примечание \\
\hline
22.05.19 & Излишняя реализация функции для проверки слова на наличие знаков нижнего регистра & Использован предикат some & \\
\hline
28.05.19 & Функция lower-case-p гарантированно работает только для
латинских букв & Добавлена функция russian-lower-case-p для корректной работы с русскими буквами & \\
\hline
\end{tabular}

\section{Замечания автора по существу работы}
Данная лабораторная работа далась мне с трудом, так как я до сих пор неосознанно переключаюсь на язык C++.

\section{Выводы}
При выполнении работы пришлось написать много излишнего кода: чтобы посмотреть, как работает множество новых для меня конструкций, исправить неизящно написанные и неработающие функции. В очередной раз нахожу язык Common Lisp неудобным для меня, порой даже сложным. Возникает вопрос, обосновано ли данное неудобство. Пока что определенного ответа у меня нет.
\end{document}