\documentclass[12pt]{article}

\usepackage{fullpage}
\usepackage{multicol,multirow}
\usepackage{tabularx}
\usepackage{ulem}
\usepackage[utf8]{inputenc}
\usepackage[russian]{babel}
\usepackage{amsmath}
\usepackage{amssymb}
\usepackage{listing}

\usepackage{titlesec}

\titleformat{\section}
  {\normalfont\Large\bfseries}{\thesection.}{0.3em}{}

\titleformat{\subsection}
  {\normalfont\large\bfseries}{\thesubsection.}{0.3em}{}

\titlespacing{\section}{0pt}{*2}{*2}
\titlespacing{\subsection}{0pt}{*1}{*1}
\titlespacing{\subsubsection}{0pt}{*0}{*0}
\usepackage{listings}
\lstloadlanguages{Lisp}
\lstset{extendedchars=false,
	breaklines=true,
	breakatwhitespace=true,
	keepspaces = true,
	tabsize=2
}
\begin{document}


\section*{Отчет по лабораторной работе №\,1 
по курсу \guillemotleft  Функциональное программирование\guillemotright}
\begin{flushright}
Студент группы М8О-307 МАИ \textit{Безлуцкая Елизавета}, \textnumero 2 по списку \\
\makebox[7cm]{Контакты: {\tt lizabezlutskaya@gmail.com} \hfill} \\
\makebox[7cm]{Работа выполнена: 17.03.2019 \hfill} \\
\ \\
Преподаватель: Иванов Дмитрий Анатольевич, доц. каф. 806 \\
\makebox[7cm]{Отчет сдан: \hfill} \\
\makebox[7cm]{Итоговая оценка: \hfill} \\
\makebox[7cm]{Подпись преподавателя: \hfill} \\

\end{flushright}

\section{Тема работы}
Примитивные функции и особые операторы Коммон Лисп.

\section{Цель работы}
Научиться вводить S-выражения в Лисп-систему, определять переменные и функции, работать с условными операторами, работать с числами, использую схему линейной и древовидной рекурсии.

\section{Задание (вариант №40)}
Запрограммируйте на языке Коммон Лисп функционал {\tt product}. Функционал должен вычислять произведение чисел от {\tt a} до {\tt b} и принимать в качестве входных параметров одноместную функцию {\tt f}, а также {\tt a} и {\tt b}.

\section{Оборудование студента}
Процессор Intel Core i5-3230M 4\,@\,2.6GHz, память: 350Gb, разрядность системы: 64.

\section{Программное обеспечение}
Ubuntu 16.04 LTS, clisp compiler

\section{Идея, метод, алгоритм}
Функция {\tt product} рекурсивна и работает следующим образом:
\begin{itemize}
\setlength{\itemsep}{-1mm} % уменьшает расстояние между элементами списка
\item если левая граница больше, чем правая, функция вернет 1
\item если левая граница меньше, чем правая, то перемножаем левую границу и результат рекурсивного вызова функции {\tt product} с параметрами {\tt a + 1} и {\tt b}, как только встретим ситуацию, когда {\tt a} равно {\tt b}, то
\item вернуть {\tt f} с параметром {\tt b}.
\end{itemize}

\section{Сценарий выполнения работы}

\section{Распечатка программы и её результаты}

\subsection{Исходный код}

\begin{lstlisting}
(defun product (f a b)
  (cond ((= a b) (funcall f b))
        ((> a b) 1)
        ((< a b)(* (funcall f a) (product f (1+ a) b)))))
\end{lstlisting}

\subsection{Результаты работы}
\begin{lstlisting}
(print (product #'1+ -1 4))
0
(print (product #'1+ 1 4))
120
(print (product #'1+ 4 4))
5
(print (product #'1+ 6 4))
1
(print (product #'1+ -5 2))
0
(print (product #'1+ -7 -3))
-720
(print (product #'abs -7 -3))
2520
\end{lstlisting}

\section{Дневник отладки}
\begin{tabular}{|c|p{5cm}|p{5cm}|c|}
\hline
Дата & Событие & Действие по исправлению & Примечание \\
\hline
17.03.19 & В ситуации, когда левая граница больше правой, функция возвращала 1 вместо 0 & Конструкция if заменена на cond с целью добавления условия <<левая граница равна правой>>, при выполнении которого возвращается правая граница, обработанная функцией {\tt f}. А при выполнении услоия <<левая граница больше, чем правая>>, функция возвращает 1. & \\
\hline
18.03.19 & Произведение нулевого количество сомножителей трактуется как 1. & При попадании в ветвь с условием {\tt a} больше {\tt b}, выходим с результатом 1. & \\
\hline
\end{tabular}

\section{Замечания автора по существу работы}
Подозреваю, я не первая, кому было так трудно разобраться в огромном количестве скобок. Если бы не данная особенность языка, программировать на языке Lisp мне было бы приятнее.

\section{Выводы}
Большую часть времени я потратила на то, чтобы изучить синтаксис языка, вспомнить необходимые мне конструкции и отключиться от языка C++. Чтобы решить задачу, сначала я нарисовала дерево рекурсии и быстро написала решение, которое мне пришло в голову. Затем придумала тесты, которые потенциально могли сломать его, и привела код в финальный вид.
Думаю, ничего сложного.

\end{document}\grid
\grid
