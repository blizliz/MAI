\documentclass[12pt]{article}

\usepackage{fullpage}
\usepackage{multicol,multirow}
\usepackage{tabularx}
\usepackage{ulem}
\usepackage[utf8]{inputenc}
\usepackage[russian]{babel}
\usepackage{listings}
\usepackage{hyperref}
\usepackage{graphicx}
\usepackage{amsmath}
\usepackage{xcolor}
\DeclareGraphicsExtensions{.png}


\begin{document}

\section*{Лабораторная работа №\,5 по курсу криптографии}

Выполнила студентка группы М8О-307Б \textit{Безлуцкая Елизавета}.

\subsection*{Условие}
Сравнить:
\begin{enumerate}
\item два осмысленных текста на естественном языке
\item осмысленный текст и текст
из случайных букв
\item осмысленный текст и текст из случайных слов
\item два текста из
случайных букв
\item два текста из случайных слов
\end{enumerate}
\noindent
Считать процент совпадения букв в сравниваемых текстах – получить дробное значение от 0
до 1 как результат деления количества совпадений на общее число букв. Расписать подробно
в отчёте алгоритм сравнения и приложить сравниваемые тексты в отчёте хотя бы для одного
запуска по всем пяти случаям. Осознать какие значения получаются в этих пяти случаях.
Привести соображения о том почему так происходит.\\
\\
Длина сравниваемых текстов должна совпадать. Привести соображения о том какой длины
текста должно быть достаточно для корректного сравнения.

\subsection*{Метод решения}
Я взяла осмысленные тексты из \href{http://www.gutenberg.org}{http://www.gutenberg.org}. Тексты из случайных букв генерировались с использованием регистрозависимого латинского алфавита. Случайные слова были взяты из файлов \href{https://github.com/first20hours/google-10000-english}{https://github.com/first20hours/google-10000-english}.\\
\\
Длина слов для текстов из случайных букв составляет от 3 до 10 символов, для текстов из слов -- беру слова из трех файлов(короткие, средние, длинные).\\
\\
Сравнение текстов происходит побуквенно, если буквы в одинаковых позициях совпали, то увеличиваем счетчик совпадений. 
\\
Результаты сравнения:\\

\begin{lstlisting}
Comparison 1: two meaningful text in natural language
Text length: 717618
Match percentage: 0.062095711088629324
Comparison 2: meaningful text and text from random letters
Text length: 717618
Match percentage: 0.035779481562614096
Comparison 3: meaningful text and text from random words
Text length: 717618
Match percentage: 0.06200234665239723
Comparison 4: two texts from random letters
Text length: 700000
Match percentage: 0.03456957142857143
Comparison 5: two texts from random words
Text length: 700000
Match percentage: 0.06566414285714287
\end{lstlisting}

\subsection*{Исходный код}
\definecolor{codegreen}{rgb}{0,0.6,0}
\definecolor{codegray}{rgb}{0.5,0.5,0.5}
\definecolor{codepurple}{rgb}{0.58,0,0.82}
\definecolor{backcolour}{rgb}{0.95,0.95,0.92}
 
\lstdefinestyle{mystyle}{
    backgroundcolor=\color{backcolour},   
    commentstyle=\color{codegreen},
    keywordstyle=\color{magenta},
    numberstyle=\tiny\color{codegray},
    stringstyle=\color{codepurple},
    basicstyle=\footnotesize,
    breakatwhitespace=false,         
    breaklines=true,                 
    captionpos=b,                    
    keepspaces=true,                 
    numbers=left,                    
    numbersep=5pt,                  
    showspaces=false,                
    showstringspaces=false,
    showtabs=false,                  
    tabsize=2
}
 
\lstset{style=mystyle}
 
\lstinputlisting[language=Python]{code/main.py}

\subsection*{Выводы}
По результатам видно, что лучше всего совпали осмысленные тексты, осмысленный текст и текст из случайных слов, а также два текста из случайных слов.\\
\\
Что касается осмысленных текстов, то здесь вероятность высокого совпадения выше по причине лингвистических особенностей. Устоявшиеся конструкции, так называемые n-граммы, часто встречающиеся слоги и т.д. Для опыта я взяла разные произведения - <<Гордость и предубеждение>> Д.Остин и <<Война и мир>> Л.Толстова. Процент совпадения получился около 0.06. Затем для интереса были взяты произведения одного автора - сказки братьев Гримм. В таком сравнении процент совпадения текстов возрос и составил около 0.07. Очевидно, что у каждого автора есть свой почерк, свой словарь, что увеличивает "повторения".\\
\\
В текстах из случайных слов в моем случае был взят единый словарь. Это, конечно же, дало высокий показатель совпадений. В случае использования разных словарей, сравнение дает более низкий результат.\\
\\
Со случайными буквами всё гораздо сложнее. Невозможно дать точную оценку совпадений, так как в тестах использовался регистрозависимый алфавит. В случае сравнения двух текстов, мы видим, что вероятность встретить ту или иную букву составила $\dfrac{1}{58}$ вместо $\dfrac{1}{26}$ в регистронезависимом алфавите. Становится ясно, что это ухудшает ситуацию.\\
\\
Попытки сравнить осмысленный текст и текст из случайных букв видятся мне не самыми удачными по причине того, что в тексте какого-либо произвидения, например, встречаются ещё и знаки препинания. Если принебречь заглавными буквами в случайном тексте, то, возможно, в сравнении будет больше смысла.

\end{document}\grid
